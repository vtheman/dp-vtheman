%=========================================================================
% (c) Michal Bidlo, Bohuslav Křena, 2008

\chapter{Introduction}
Human is a social creature and likes to chat, share feelings and ideas. At first we managed to do so by making simple sounds. Those sounds later on developed into words. Then much later the human race started to feel the need to record what we were thinking. We made up symbols and started to write. As the society grew and spread, we wanted to communicate with people from other tribes and villages. At first we would travel and use spoken words, but as the distances grew we figured we can have our thoughts delivered in writing. Mail was born. In 1844 telegraph was invented by Samuel Morse followed by telepthone in 1874 by Alexander Graham Bell. And finally in 1969 the Internet was created. All of these inventions aimed to provide means of communication to satisfy the needs of the evolving society.  

In the early days of the Internet email was the main means of communication. And just like regular mail people would have their electronic mailboxes to which the emails were delivered. Email was a huge step forward for it provided a way to almost instantly deliver text from one place to another regardless of the distance for free. The main disadvantage of email is that people had to check their mailboxes read new mail and then reply. It is just neither fast nor convenient enough for team cooperation when team members are far apart. For those and other purposes like chatting Instant Messenger programs were introduced.    

An IM program offers realtime communication between two people via text messages that are delivered from one user to another instantly. Instant messangers became very popular and started adding on features like multiuser chat, various games and most importantly VoIP(Voice over IP) support. VoIP capable IM like skype have become extremely popular at first for making it possible for people to call each other for free over the internet. Later video conferencing capability was added, so you could talk and see you colleague at the same time. One more thing comes in extremely handy when working in a team - a whiteboard.

At this time there is no usable IM providing VoIP and shared whiteboard for GNU/Linux. This thesis aims to add VoIP support to an existing XMPP client with shared board called Makneto. Makneto was created by Jaroslav Řezník as a master's thesis in 2008. At this point it is using iris library for XMPP communication. The shared board data is also transferred over XMPP. One of the goals of this thesis is to port Makneto to telepathy, which is now a very reliable and robust library for communication for numerous protocols.  

The following chapter gives a detailed description of the current version of program Makneto. Chapter number three will focus on XMPP (ro telepathy)
Description of all thesis' chapters.



\chapter{Makneto}

\chapter{XMPP}

\chapter{Telepathy}

\chapter{Audio and video streaming protocols}

\chapter{Implementation}

\chapter{Conclusion}
Závěrečná kapitola obsahuje zhodnocení dosažených výsledků se zvlášť vyznačeným vlastním přínosem studenta. Povinně se zde objeví i zhodnocení z pohledu dalšího vývoje projektu, student uvede náměty vycházející ze zkušeností s řešeným projektem a uvede rovněž návaznosti na právě dokončené projekty.

%=========================================================================
