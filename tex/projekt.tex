%============================================================================
% tento soubor pouzijte jako zaklad
% (c) 2008 Michal Bidlo
% E-mail: bidlom AT fit vutbr cz
%============================================================================
% kodovaní: iso-8859-2 (zmena prikazem iconv, recode nebo cstocs)
%----------------------------------------------------------------------------
% zpracování: make, make pdf, make desky, make clean
% p?ipomínky posílejte na e-mail: bidlom AT fit.vutbr.cz
% vim: set syntax=tex:
%============================================================================
\documentclass[english]{fitthesis} % odevzdani do wisu - odkazy, na ktere se da klikat
%\documentclass[cover,print]{fitthesis} % pro tisk - na odkazy se neda klikat
%\documentclass[english,print]{fitthesis} % pro tisk - na odkazy se neda klikat
%      \documentclass[english]{fitthesis}
% * Je-li prace psana v anglickem jazyce, je zapotrebi u tridy pouzit 
%   parametr english nasledovne:
%\documentclass[cover][english]{fitthesis}
% * Neprejete-li si vysazet na prvni strane dokumentu desky, zruste 
%   parametr cover

% zde zvolime kodovani, ve kterem je napsan text prace
% "latin2" pro iso8859-2 nebo "cp1250" pro windows-1250, "utf8" pro "utf-8"
%\usepackage{ucs}
\usepackage[utf8]{inputenc}
\usepackage[T1, IL2]{fontenc}
\usepackage{url}
\DeclareUrlCommand\url{\def\UrlLeft{<}\def\UrlRight{>} \urlstyle{tt}}

%zde muzeme vlozit vlastni balicky
\usepackage{lineno}
\usepackage{color}
%definice lstlisting
\usepackage{listings}

\def\lstlistingname{Example}
\definecolor{gray}{rgb}{0.4,0.4,0.4}
\definecolor{darkblue}{rgb}{0.0,0.0,0.6}
\definecolor{cyan}{rgb}{0.0,0.6,0.6}
\lstdefinelanguage{XML}
{
 morestring=[b]",
 morestring=[d]{>}{<},
 morecomment=[s]{<?}{?>},
 stringstyle=\color{black},
 identifierstyle=\color{darkblue},
 keywordstyle=\color{cyan},
 morekeywords={xmlns,version,type, message, from, to, body, iq, stream, mechanism, presence, status,xml:lang, item, jid, query,show}% list your attributes here
}
\lstset{
  basicstyle=\ttfamily,
  columns=fullflexible,
  showstringspaces=false,
  commentstyle=\color{gray}\upshape
}
\lstset{ %
language=XML, % the language of the code
%basicstyle=\footnotesize, % the size of the fonts that are used for the code
numbers=left, % where to put the line-numbers
numberstyle=\footnotesize, % the size of the fonts that are used for the line-numbers
stepnumber=1, % the step between two line-numbers. If it's 1, each line will be numbered
numbersep=10pt, % how far the line-numbers are from the code
backgroundcolor=\color{white}, % choose the background color. You must add \usepackage{color}
showspaces=false, % show spaces adding particular underscores
showstringspaces=false, % underline spaces within strings
showtabs=false, % show tabs within strings adding particular underscores
frame=tb, % adds a frame around the code
tabsize=2, % sets default tabsize to 2 spaces
captionpos=b, % sets the caption-position to bottom
breaklines=true, % sets automatic line breaking
breakatwhitespace=false, % sets if automatic breaks should only happen at whitespace
title=\lstname, % show the filename of files included with \lstinputlisting; also try caption instead of title
escapeinside={\%*}{*)}, % if you want to add a comment within your code
morekeywords={*,...}, % if you want to add more keywords to the set
commentstyle=\color{gray}\upshape
}

% =======================================================================
% balí?ek "hyperref" vytvá?í klikací odkazy v pdf, pokud tedy pou?ijeme pdflatex
% problém je, ?e balí?ek hyperref musí být uveden jako poslední, tak?e nem??e
% být v ?ablon?
\ifWis
\ifx\pdfoutput\undefined % nejedeme pod pdflatexem
\else
  \usepackage{color}
  \usepackage[unicode,colorlinks,hyperindex,plainpages=false,pdftex]{hyperref}
  \definecolor{links}{rgb}{0.4,0.5,0}
  \definecolor{anchors}{rgb}{1,0,0}
  \def\AnchorColor{anchors}
  \def\LinkColor{links}
  \def\pdfBorderAttrs{/Border [0 0 0] }  % bez okraj? kolem odkaz?
  \pdfcompresslevel=9
\fi
\fi

%Informace o praci/projektu
%---------------------------------------------------------------------------
\projectinfo{
  %Prace
  project=SP,            %typ prace BP/SP/DP/DR
  year=2011,             %rok
  date=\today,           %datum odevzdani
  %Nazev prace
  title.cs={VoIP v jabber klientu},  %nazev prace v cestine
  title.en={VoIP in jabber client}, %nazev prace v anglictine
  %Autor
  author={Vojtěch Kulička},   %jmeno prijmeni autora
  author.title.p=Bc., %titul pred jmenem (nepovinne)
  %author.title.a=PhD, %titul za jmenem (nepovinne)
  %Ustav
  department=UPGM, % doplnte prislusnou zkratku: UPSY/UIFS/UITS/UPGM
  %Skolitel
  supervisor={Jozef Mlích}, %jmeno prijmeni skolitele
  supervisor.title.p={Ing.},   %titul pred jmenem (nepovinne)
  %supervisor.title.a={},    %titul za jmenem (nepovinne)
  %Klicova slova, abstrakty, prohlaseni a podekovani je mozne definovat 
  %bud pomoci nasledujicich parametru nebo pomoci vyhrazenych maker (viz dale)
  %===========================================================================
  %Klicova slova
  keywords.cs={VoIP, IM, sdílená tabule, XMPP, telepathy}, %klicova slova v ceskem jazyce
  keywords.en={VoIP, IM, Shared whiteboard, XMPP, telepathy}, %klicova slova v anglickem jazyce
  %Abstract
  abstract.cs={Práce se zabývá možnostmi přidání funkcionality do existujícího XMPP programu se sdílenou tabulí. Analyzuje možnosti využití současných technologií pro podporu VoIP. Cílem je port klienta na komunikační architekturu telepathy a implementace VoIP.}, % abstrakt v ceskem jazyce
  abstract.en={This thesis tackles the issues of implementing a VoIP support into an XMPP based IM application. The state of the art is analyzed to find a suitable technology to base the VoIP on. The work's goal is to port the existing client application to network framework telepathy and implentation of VoIP.}, % abstrakt v anglickem jazyce
  %Prohlaseni
  declaration={Prohlašuji, že jsem tuto diplomovou práci vypracoval samostatně pod vedením pana inženýra Jozefa Mlícha},
  %Podekovani (nepovinne)
  acknowledgment={Děkuji svému vedoucímu Ing.Jozefu Mlíchovi a Ing.Jaroslavu Řezníkovi za odbornou pomoc. } % nepovinne
}

%Abstrakt (cesky, anglicky)
%\abstract[cs]{Do tohoto odstavce bude zapsán výtah (abstrakt) práce v ?eském jazyce.}
%\abstract[en]{Do tohoto odstavce bude zapsán výtah (abstrakt) práce v anglickém jazyce.}

%Klicova slova (cesky, anglicky)
%\keywords[cs]{Sem budou zapsána jednotlivá klí?ová slova v ?eském jazyce, odd?lená ?árkami.}
%\keywords[en]{Sem budou zapsána jednotlivá klí?ová slova v anglickém jazyce, odd?lená ?árkami.}

%Prohlaseni
%\declaration{Prohla?uji, ?e jsem tuto bakalá?skou práci vypracoval samostatn? pod vedením pana X...
%Dal?í informace mi poskytli...
%Uvedl jsem v?echny literární prameny a publikace, ze kterých jsem ?erpal.}

%Podekovani (nepovinne)
%\acknowledgment{V této sekci je možno uvést poděkování vedoucímu práce a těm, kteří poskytli odbornou pomoc
%(externí zadavatel, konzultant, apod.).}

\begin{document}
  % Vysazeni titulnich stran
  % ----------------------------------------------
  \maketitle
  % Obsah
  % ----------------------------------------------
  \tableofcontents
  
  % Seznam obrazku a tabulek (pokud prace obsahuje velke mnozstvi obrazku, tak se to hodi)
  % \listoffigures
  % \listoftables 

  % Text prace
  % ----------------------------------------------
  %=========================================================================
% (c) Michal Bidlo, Bohuslav Křena, 2008

\chapter{Introduction}
Human is a social creature and likes to chat, share feelings and ideas. At first we managed to do so by making simple sounds. Those sounds later on developed into words. Then much later the human race started to feel the need to record what we were thinking. We made up symbols and started to write. As the society grew and spread, we wanted to communicate with people from other tribes and villages. At first we would travel and use spoken words, but as the distances grew we figured we can have our thoughts delivered in writing. Mail was born. In 1844 telegraph was invented by Samuel Morse followed by telepthone in 1874 by Alexander Graham Bell. And finally in 1969 the Internet was created. All of these inventions aimed to provide means of communication to satisfy the needs of the evolving society.  

In the early days of the Internet email was the main means of communication. And just like regular mail people would have their electronic mailboxes to which the emails were delivered. Email was a huge step forward for it provided a way to almost instantly deliver text from one place to another regardless of the distance for free. The main disadvantage of email is that people had to check their mailboxes read new mail and then reply. It is just neither fast nor convenient enough for team cooperation when team members are far apart. For those and other purposes like chatting Instant Messenger programs were introduced.    

An IM program offers realtime communication between two people via text messages that are delivered from one user to another instantly. Instant messangers became very popular and started adding on features like multiuser chat, various games and most importantly VoIP(Voice over IP) support. VoIP capable IM like skype have become extremely popular at first for making it possible for people to call each other for free over the internet. Later video conferencing capability was added, so you could talk and see you colleague at the same time. One more thing comes in extremely handy when working in a team - a whiteboard.

At this time there is no usable IM providing VoIP and shared whiteboard for GNU/Linux. This thesis aims to add VoIP support to an existing XMPP client with shared board called Makneto. Makneto was created by Jaroslav Řezník as a master's thesis in 2008. At this point it is using iris library for XMPP communication. The shared board data is also transferred over XMPP. One of the goals of this thesis is to port Makneto to telepathy, which is now a very reliable and robust library for communication for numerous protocols.  

The following chapter gives a detailed description of the current version of program Makneto. Chapter number three will focus on XMPP (ro telepathy)
Description of all thesis' chapters.



\chapter{Makneto}

\chapter{XMPP}

\chapter{Telepathy}

\chapter{Audio and video streaming protocols}

\chapter{Implementation}

\chapter{Conclusion}
Závěrečná kapitola obsahuje zhodnocení dosažených výsledků se zvlášť vyznačeným vlastním přínosem studenta. Povinně se zde objeví i zhodnocení z pohledu dalšího vývoje projektu, student uvede náměty vycházející ze zkušeností s řešeným projektem a uvede rovněž návaznosti na právě dokončené projekty.

%=========================================================================
 % viz. obsah.tex

  % Pouzita literatura
  % ----------------------------------------------
\ifczech
  \bibliographystyle{czechiso}
\else 
  \bibliographystyle{plain}
%  \bibliographystyle{alpha}
\fi
  \begin{flushleft}
  \bibliography{literatura} % viz. literatura.bib
  \end{flushleft}
  \appendix
  
  \chapter{CD Content}
\textbf{/Makneto/*} - revision of program Makneto from May 23rd 2011\newline
\textbf{/Makneto/README} - manual for compiling Makneto\newline
\textbf{/thesis/*} - source code of this thesis\newline
\textbf{/projekt.pdf} - text of this work\newline
%\chapter{Dependencies}
%\chapter{Konfigrační soubor}
%\chapter{RelaxNG Schéma konfiguračního soboru}
%\chapter{Plakat}

 % viz. prilohy.tex
\end{document}
